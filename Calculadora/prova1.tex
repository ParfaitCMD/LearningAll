\documentclass{article}
\usepackage{fontspec}

% Define a fonte principal como Arial
\setmainfont{Arial}

\begin{document}

\section*{Revisão - Engenharia de Software}

\textbf{Garantia da Qualidade:} O objetivo dessa etapa é definir e implementar procedimentos e padrões de desenvolvimento. A ideia é criar um ambiente onde o software de alta qualidade seja a \textbf{regra}, e não a exceção. Pense nisso como a parte preventiva: você estabelece as ``regras do jogo'' antes mesmo de começar a programar, para garantir que o resultado final será bom.

---

\textbf{Planejamento da Qualidade:} Aqui, você cria um plano detalhado para garantir a qualidade de um projeto específico. Essa etapa é a ``tradução'' dos padrões de garantia da qualidade para um projeto real. Você decide, por exemplo, quais testes serão feitos, quem será responsável por eles e quais métricas serão usadas para medir a qualidade do produto. É a fase de \textbf{organização} do processo de qualidade para aquele projeto em particular.

---

\textbf{Controle da Qualidade:} Esta é a parte de \textbf{execução} e verificação. Depois que o software foi desenvolvido, o controle de qualidade garante que ele realmente segue o que foi planejado. Essa atividade envolve a checagem e o monitoramento do processo para ter certeza de que tudo foi feito conforme os padrões e o plano estabelecido. É a etapa onde você ``coloca a mão na massa'' para verificar se o produto final atende aos requisitos de qualidade.

\end{document}